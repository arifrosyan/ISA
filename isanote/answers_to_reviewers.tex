\documentclass[a4paper]{article}
\usepackage{url}
\usepackage{natbib}

\newcommand{\Rpackage}[1]{`\texttt{#1}'}
\newcommand{\Rfunction}[1]{`\texttt{#1}'}
\newcommand{\Rclass}[1]{\textsl{#1}}

\setlength{\parskip}{12pt}
\setlength{\parindent}{0pt}

\title{Modular analysis of gene expression data with R\\
  Answers to reviewers}
\author{G\'abor Cs\'ardi, Zolt\'an Kutalik and Sven Bergmann}

\begin{document}

\maketitle

First of all, we want to thank the reviewers for their thorough work
and useful suggestions. In the following, the reviewers' comments are
indented as a block.

\section{Reviewer 1}

\begin{quote}
Comments to the Author

This Application Note presents a software package that implements the
Iterative Signature Algorithm and provides a convenient visualization
of it’s results. The method is useful and the software may be of help
to interested users. 
\end{quote}

\begin{quote}
I have two problems with the Note in its present form.
The first may be due to my misunderstanding of the method. My
understanding was that some threshold Tg is applied on the genes and
another one, Tc, on the conditions. The aim is to reach a bicluster (a
set of conditions C and a set of genes G) such that G contains only
genes with expression above Tg for all the conditions of C, and
conditions are included in C only if the expression of all genes of G
exceeds Tc (one can replace a lower threshold with an upper one). The
heatmap of the Figure implies that my understanding was incorrect –
the conditions satisfy a double threshold (expressions are either
above or below some thresholds) and the genes have expressions above a
Tg in some conditions or below some other Tg’ in other conditions. 
I urge the authors to define what specifies or defines a bicluster –
do it briefly and sacrifice a few lines from elsewhere. 
\end{quote}

Indeed, by default the software uses a double threshold. This can be
changed, as explained in the reference manual of the ISA functions,
and the ISA tutorial titled `The Iterative Signature Algorithm for
Gene Expression Data'.

To  clarify things better, we have added the following to the main
text:
\textsl{The output of ISA is a collection of potentially overlapping
modules. Every module contains genes, that are over- and/or
under-expressed, in samples that belong to the module.
In every module, each gene and each sample is attributed a
score between minus one and one, that reflects the strength of the
association with the module. Moreover, if the scores of two genes of a
module have the same sign, then they are correlated (across the
samples of the module), opposite signs mean
anti-correlation. Similarly, if two sample
scores have the same sign, then these samples are correlated (across
the genes of the module), opposite signs indicate anti-correlation.}

The tutorial on the ISA homepage explains the ISA iteration in detail:
\textsl{The thresholding is an important step of the ISA, without
  thresholding ISA would be equivalent to a (not too effective)
  numerical singular value decomposition (SVD) algorithm. Currently
  thresholding is done by calculating the mean and standard deviation
  of the vector and keeping only elements that are further than a
  given number of standard deviations from the mean. Using the
  ``direction'' parameter, one can keep values that are
  (a) significantly higher (``up''); (b) lower (``down'') than the
  mean; or (c) both (``updown'').}

\begin{quote}
My second concern is minor – the abstract implies that the ISA is
special in allowing overlapping cluster assignments. This is incorrect
– there are other biclustering methods that do this. If I am not
mistaken, all three original biclustering methods that came out in
2000 (Cheng, Getz and Califano) allow this. 
\end{quote}

We have removed the emphasis on the overlapping biclusters of the
ISA and added references to these methods.

\section{Reviewer 2}

\begin{quote}
Comments to the Author

Summary

This manuscript describes two software packages for GNU R, for
discovering biclusters (here called transcription modules. see \citet{tanay05} for
a survey) in a given gene expression matrix based on the Iterative
Signature Algorithm (ISA) \citep{isa}. The first package, called isa2,
contains the implementation of the basic ISA, and can be used to
analyze any tabular data. The second package, called eisa, is built on
isa2. It adds support to standard BioConductor \citep{BioC} data structures,
and contains gene expression specific tools for visualization of
biclusters. Both packages support execution of ISA with default
parameters, and also execution of ISA step by step with user-specified
parameters. The steps of ISA are normalization, iteration, filtering
of modules and enrichment calculation. In the current note, the
authors present both packages, and describe how to install and run
them step by step, using the acute lymphoblastic leukemia data set
(ALL) \citep{chiaretti04} as an example. 
\end{quote}

\begin{quote}
Overall evaluation

The manuscript and its accompanying tutorials are well written and
give an adequate theoretical background to ISA. The tutorials include
many detailed and visualized examples that make learning the package
quite easy for an R user. However, the following major points should
be taken into consideration: 
\end{quote}

\begin{quote}
1.      The manuscript and the website lack explanations about how the
authors implemented ISA and how they used the advantages of R. The
authors emphasize that the packages are highly optimized and run
extremely fast even for very large datasets.  Therefore, the
accompanying material should contain a summary describing how this
efficiency was achieved. In addition, a graph showing execution time
as a function of dataset size should be given. 
\end{quote}

We have added a new document to the ISA homepage at
\url{http://www.unil/ch/cbg/ISA}, titled `ISA internals'. This
document contains the details of our implementations, and also
execution time analysis.

\begin{quote}
2.      The tutorials give examples of execution on a single rather
old dataset (ALL) which is quite small in comparison to datasets of
recent years. Here the authors should give results for ISA execution
on more up-to-date datasets, and on heterogeneous datasets containing
profiles from multiple studies. 
\end{quote}

We believe that the tutorials are best if they can be run on an
average desktop computer, interactively, without spending much
effort on collecting and assembling the data. This is why we chose the
ALL dataset. It is included in BioConductor, and used in the tutorials
of many BioConductor packages. There is a high chance that it is
already installed on the user's machine, and if it is not, then it can
be downloaded easily and quickly.

We wrote a second tutorial that uses a bigger, more recent data set, 
titled `Tissue specific expression with the Iterative Signature
Algorithm', please see the ISA homepage at
\url{http://www.unil.ch/cbg/ISA}. The `ISA internals' document uses a
new, larger data set, as well.

Of course, the ISA can be used for much larger data sets (see
e.g. \cite{bergmann04}), but then the analysis takes at least hours
instead of minutes.

\begin{quote}
To summarize, the software is well motivated and enables users of the
R environment to utilize ISA. I suggest accepting the manuscript
conditional on solving the issues mentioned in this report. 
\end{quote}

\begin{quote}
Additional important remarks:

1.      The manuscript should contain references to additional
biclustering implementations, in R or in other environments, so that
the user could choose the one that best fits his needs in terms of
input/output, species, etc. Examples are SAMBA (via the Expander
software) \citep{sharan02,tanay04} and BicAt \citep{barkow06}. 
\end{quote}

We have added these references.

\begin{quote}
2.      The sample score plots example given in the EISA tutorial
(\url{http://www2.unil.ch/cbg/index.php?title=EISA\_tutorial}, section 6.5)
is not sufficiently clear. What are the scores represented by the
horizontal lines and how can one calculate them? Moreover, how can one
separate B-cell and T-cell leukemia samples using this plot? 
\end{quote}

We have also explained sample score plots better in the tutorial. 
Please note, that the tutorials do not aim to explain all details 
about the ISA functions; the reference manual, that is included in the
package, is the place for that.

The module plotted in section 6.5 indeed cannot be used to separate
B-cell and T-cell samples, we point this out in the text.

\begin{quote}
3.      It is not quite clear what is the difference between the ISA
implementation included in the isa2 package and the ISA implementation
included in the eisa package. 
\end{quote}

We have explained the difference between the two packages in the newly
added `ISA internals' document, section `Why two packages?'.

\begin{quote}
4.      It should be clarified who are the potential users for the ISA
implementation for R environment. Assuming these are mainly
biologists, bioinformaticians and biostatisticians (and not only
programmers), more technical details of how to work with the packages
should be given.
\end{quote}

We have added the following to the abstract:
\textsl{
Potential users for these packages are all R and
BioConductor users dealing with tabular (e.g. gene expression) data.
}

\begin{quote}
The following points are important examples: 
a.      Since the ISA implementation in eisa package imports data from
another package (e.g. hgu95av2.db), a tutorial for how to create such
a .db package from a given expression data text file should be given
(or referred to). 
\end{quote}

Users analyzing gene expression data almost never need to build
these annotation packages, they are included in BioConductor, for all
standard chips. Only for custom chips might be useful to create such
packages by hand; but even in this case, they are not required, the
user can create ExpressionSet objects that do not contain chip
annotation, or use the \Rpackage{isa2} package.

\begin{quote}
b.      Since both ISA packages use various data structures, a short
explanation of how to construct and use them should be given (or
references to such explanation), and also how to export data from
these structures to external files. 
\end{quote}

For the \Rpackage{isa2} package, this is included in the reference
manual. For the \Rpackage{eisa} package, the \Rclass{ISAModules} class
is an opaque data type, the representation of it might change in the
future, and the user should not rely on its current
definition. Instead, various functions are provided that deal with
these kind of objects, they are documented on the manual page of the
\Rclass{ISAModules} class, in the reference manual.

\begin{quote}
Minor remarks:

1.      The authors describe methods that perform enrichment tests for
the gene sets corresponding to the ISA modules against various
databases. In particular, they perform enrichment analysis against GO
database, KEGG pathway database, chromosomes and the TargetScan \citep{targetscan}
database. One may also want to investigate the correspondence of
biclusters and protein-protein interaction networks (which have been
derived from other types of data than gene expression data), e.g. the
DIP database \citep{salwinski04}. An example for such analysis can be
found in \citet{prelic06}.
\end{quote}

We have added new classes and functions to perform enrichment analysis
on the ISA modules, against any gene label or database that the user
provides. These functions are explained in the manual page of the
\Rfunction{ISAEnrichment} function.

\begin{quote}
2.      The heatmap image of a transcription module obtained by any
one of the two methods presented in the tutorial should include a
color legend. 
\end{quote}

We have added the color legend to the first heatmap. We removed the 
second heatmap from the tutorial entirely, as adding a heatmap to 
it would be too cumbersome for the user.

\begin{quote}
3.      Since ISA function accepts an ExpressionSet, one must obtain
annotations for its expression data file in order to run ISA function
on it. Annotations should not be a prerequisite for running ISA. 
\end{quote}

Annotations are not required to run ISA. First, the purpose of the
\Rpackage{isa2} package is to run ISA on any tabular data set,
biological or not. As an alternative, the user can construct an
\Rclass{ExpressionSet} object without annotation, and this can be used
with the function of the \Rpackage{eisa} package.

\begin{quote}
4.      In section 10.8 it is mentioned that two significant modules
were found using random seeds and two significant (separating) modules
were found using smart seeds. However, in figure 7 it is noted that
only one module found using smart seeds. At the end of the section it
is mistakenly written "As it turns out, one modules found". 
\end{quote}

We have corrected this.

\bibliographystyle{natbib}
\bibliography{isa}

\end{document}
